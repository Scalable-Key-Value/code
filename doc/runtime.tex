%%%%%%%%%%%%%%%%%%%%%%%%%%%%%%%%%%%%%%%%
% Copyright (c) IBM Corp. 2014
% All rights reserved. This program and the accompanying materials
% are made available under the terms of the Eclipse Public License v1.0
% which accompanies this distribution, and is available at
% http://www.eclipse.org/legal/epl-v10.html
%
% Contributors:
%    lschneid
%%%%%%%%%%%%%%%%%%%%%%%%%%%%%%%%%%%%%%%%

\section{Configuration}\label{sec:skv:config}
This section will cover the compiletime and runtime configuration of
SKV client and server.

\subsection{Compile Time Configuration}\label{sec:skv:config:cmp}
There are a ton of macros that affect the behaviour and performance of
SKV.  Most of them are bundled in \verb|make.conf.in|.

When SKV source is build for the first time, the \verb|bootstrap.sh|
script will be executed.  It creates an initial \verb|make.conf| by
copying the original \verb|make.conf.in|.  After this step the default
configuration is set up.

For more system and environment specific configurations like paths and
compiler flags, SKV provides an option to do individual configuration
without modifying the main config.  You'll find a few files named
\verb|make.conf_dev| or \verb|make.conf_rpm|.  These files are
prepended to \verb|make.conf| and selected via environment variable
\verb|SKV_CONF|.  For example:

\begin{verbatim}
$> export SKV_CONF=dev
\end{verbatim}
 
Will select \verb|make.conf_dev| to be used to compile SKV.  If any of
the existing files is close to what you plan to do, make your own copy
and set \verb|SKV_CONF| accordingly.


\paragraph{IT\_API Configuration} works the same way as the SKV
configuration.  Since the IT\_API component was moved into SKV only
recently, the build and configuration is not yet fully merged.
Therefore, it still has parts of the configuration in separate files
in the \verb|it_api| directory.



\subsection{Client Libraries}\label{sec:config:libs}
SKV supports MPI-based and non-MPI clients.  Both have some slight
differences in the API, especially for initialization.  This is also
reflected by the availability of 2 flavors of libraries that need to
be linked to a client application.

The \verb|libskv_common| and \verb|libskv_client| have a non-MPI
version and a MPI-version.  The MPI-version includes \verb|_mpi| in
the names.

To build a non-MPI client, you need to define the macro
\verb|SKV_CLIENT_UNI| before including
\verb|include/client/skv_client.hpp|.


There's also a C-style library for SKV named \verb|libskvc.a| (and
\verb|libskvc_mpi.a| for MPI clients).  The C include file is
\verb|lib/skv.h|.


\subsection{Runtime Configuration}\label{sec:skv:config:run}
The main configuration of SKV is done via the \verb|skv_server.conf|
file.  Despite its name, it's also used by the client to find out
about the server location and connection options.

An example server config file can be found in
\verb|system/bgq/etc/skv_server.conf|.  The comments in this file
explain the details of each setting.  However, for convenience,
settings are explained below:
\begin{description}
\item[SKV\_SERVER\_PORT] will be the base port number the server will
  start trying to listen.  If the port is blocked by another SKV
  server instance (\abrEG if multiple servers run on the same node),
  SKV will attempt to bind to the subsequent ports.  The limit for
  attempts is determined by the macro \verb|SKV_MAX_SERVER_PER_NODE|
  in file \verb|include/common/skv_types.hpp|.
\item[SKV\_SERVER\_READY\_FILE] determines the file name that the
  server will create when it's ready to accept connections from
  clients.  The file is an empty file.
\item[SKV\_SERVER\_MACHINE\_FILE] This file will contain the IP
  addresses/hostnames and port numbers of the running server.  This
  type of file is required for a client to connect to a server.
\item[PERSISTENT\_FILENAME] This is the base file name of the file or
  directory used by the SKV server for persisting the data.  For
  example, in case of the in-memory storage back-end, this will be the
  base name of the mmapped file.
\item[PERSISTENT\_FILE\_LOCAL\_PATH] represents the full absolute path
  and base name for the persistent file or directory.  The server will
  append a rank number and maybe other identifiers to make this file
  unique within a server group.
\item[SKV\_SERVER\_LOCAL\_INFO\_FILE] In case SKV is exclusively run
  on a single node with multiple instances, this file is used instead
  of the server machine file.  It is suggested to use the same name
  here as for \verb|SKV_SERVER_MACHINE_FILE| because this option is
  obsolete and might disappear in the future.
\item[SKV\_SERVER\_COMM\_IF] Determines the primary interface that's
  picked by the server.  The IP address or hostname of this device
  will be placed in the machine file.
\end{description}


\subsubsection{Running the Server}\label{sec:skv:run:server}
When the server is started, the following options for the config file
are available:
\begin{enumerate}
\item an absolute path and file name provided at the command line via
  \verb|-c| option.
\item A user specific config file \verb|$(HOME)/.skv_server.conf|
\item A system-wide config file \verb|/etc/skv_server.conf|
\item Compile-time defaults will be used if there's no config file
  found or provided by any of the above options.
\end{enumerate}



\subsubsection{Running a Client}\label{sec:skv:run:client}
During the initialization of a client, the configuration file is
looked up via:
\begin{enumerate}
\item A user specific config file \verb|$(HOME)/.skv_server.conf|
\item A system-wide config file \verb|/etc/skv_server.conf|
\item Compile-time defaults will be used if there's no config file
  found or provided by any of the above options.
\end{enumerate}




\endinput



%%% Local Variables: 
%%% mode: latex
%%% TeX-master: "skvdoc"
%%% End: 
