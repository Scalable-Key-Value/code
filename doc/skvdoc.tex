%%%%%%%%%%%%%%%%%%%%%%%%%%%%%%%%%%%%%%%%
% Copyright (c) IBM Corp. 2014
% All rights reserved. This program and the accompanying materials
% are made available under the terms of the Eclipse Public License v1.0
% which accompanies this distribution, and is available at
% http://www.eclipse.org/legal/epl-v10.html
%
% Contributors:
%    lschneid
%%%%%%%%%%%%%%%%%%%%%%%%%%%%%%%%%%%%%%%%

\documentclass[a4paper, 10pt]{article}

\includeonly{client_api, back_end_api, runtime}

\usepackage[a4paper]{geometry}
\usepackage{times}
\usepackage{graphicx}
\usepackage{color}
\usepackage{url}
\usepackage{todonotes}
\usepackage{skv_macros}
\usepackage{listings}

\title{SKV -- Scalable Key/Value Store}
\author{contact: Lars Schneidenbach \\
  schneidenbach@us.ibm.com
}


\lstset{language=C++,
  frame=single, 
  basicstyle=\small, 
  keywordstyle=\color{black},
  commentstyle=\color{green},
  stringstyle=\color{brown},
  breaklines=true,
  morekeywords={ibv_open_device,ibv_alloc_pd,ibv_create_cq,ibv_create_qp,ibv_modify_qp,ibv_post_send,ibv_post_recv,ibv_poll_cq,ibv_reg_mr,
    uint64_t,uint32_t,uint16_t,uint8_t,ibv_free_device_list,ibv_get_device_list,ibv_get_device_name,ibv_dereg_mr,ibv_destroy_qp,ibv_destroy_cq,
    ibv_destroy_cq,ibv_dealloc_pd,ibv_close_device},
}


\begin{document}
\maketitle
\tableofcontents

\section{SKV Overview}\label{sec:overview}
The design of the Scalable Key/Value Store (SKV) is influenced by
Berkeley DB (BDB) key-value interface and the Remote Direct Memory
Access (RDMA) communication paradigm.  

SKV is a parallel client, parallel server, key-value database system
with basic function similar to BDB. SKV storage can be drawn from main
memory of the node of the parallel machine or from a single-node
key/value store that serves as a storage back-end for the distributed
SKV.  Currently, there's a rocksdb back-end available that allows the
storage to be in the file system.

Each node contributing storage (memory) runs a parallel SKV server and
collectively are known as an SKV Server Group. The SKV client library
runs in an MPI parallel environment (a non-MPI library is available
too).  It provides an API and manages the client end of the message
protocol between the client program and the SKV server port.  SKV
clients and SKV servers operate on a request - response model.  Any
valid SKV command can be sent to any SKV server.  The SKV server group
expects to run on a homogeneous parallel machine.  The clients using
the SKV library accessing a SKV server group may run in the server
partition or externally.  There is exactly 1 server running on each
node of a machine allocation.  The SKV library provides a standard
C/C++ API similar to BDB.  Rather than store data to a file, the SKV
library uses a messaging interface to exchange data with the SKV
server group.  The SKV client-server protocol involves short control
messages handled in an RPC style exchange and bulk transfers handled
via an RDMA type exchange.  Client connection to the SKV server group
is managed so that a single connection point provides RDMA access to
all servers across the machine.


\subsection{The Partitioned Data Set}\label{sec:overview:pds}
SKV uses Partitioned Data Set (PDS) to store tables.

The PDS is the primary structure used to organize SKV data in the
storage of a SKV server group.  A PDS is a distributed container of
key-value records.  When a PDS is opened a data structure is returned
which may be shared with other processes via message passing,
broadcast for example, to enable a parallel process to rapidly gain
global access.

The SKV Server that creates a PDS is the PDS root node.  Each SKV
Server uses a unique ID to name the created PDS.  The unique id is
formed in such a way that the SKV Server address is an accessible
component of the Id.  It is up to the creator of the PDS to keep track
of the pds id including any mapping from a name space map.

A PDS is read and written on a record basis.  To access a record, a
key must be provided.  Parts of records may be read or written.
Records may be compressed.  Currently, records are distributed to
nodes based on a hash of the key, then stored in a red-black tree in
memory or handed to the rocksdb back-end.  Every key hashes to exactly
1 node.  A PDS corresponds to a database table and a record
corresponds to a database row.

PIMD client has the ability to request the server group to take a
current snapshot of the data image and save it to persistent storage
on disk at a specified location.  The server group could then be
restarted with the data from a snapshot image.


%%%%%%%%%%%%%%%%%%%%%%%%%%%%%%%%%%%%%%%%
% Copyright (c) IBM Corp. 2014
% All rights reserved. This program and the accompanying materials
% are made available under the terms of the Eclipse Public License v1.0
% which accompanies this distribution, and is available at
% http://www.eclipse.org/legal/epl-v10.html
%
% Contributors:
%    lschneid
%%%%%%%%%%%%%%%%%%%%%%%%%%%%%%%%%%%%%%%%

\section{Client Interface}\label{sec:api}
This section explains the client API in the way that applications
would use to access the SKV Server.

\subsection{Client Initialization and Exit}\label{sec:api:init}
\begin{lstlisting}
skv_status_t Init( skv_client_group_id_t aCommGroupId,
                   MPI_Comm aComm,
                   int aFlags,
                   const char* aConfigFile = NULL );
\end{lstlisting}

The first thing for every client to do is to initialize SKV.
\begin{description}
\item[aCommGroupId] This is an identifier for this group of clients.
  It helps the server to distinguish between different client groups
  and make sure data is retrieved or returned to the correct client.
  This setting is more important for multi-process client (\abrEG MPI
  based applications).
\item[aComm] The MPI communicator has to be provided for the MPI
  version of the client API (libskv\_mpi, libskv\_client\_mpi).
\item[aFlags] Initialization flags for SKV. Currently unused.
\item[aConfigFile] The path and name of the skv configuration file.
  This file contains several important settings for the server and the
  client.  The parameter is optional.  If omitted,
  \code{/etc/skv\_server.conf} will be used as a default.  See
  \ref{sec:skv:config} for details.
\end{description}
The initialization with this call can be thought of the client side
initialization of SKV.  It does not attempt to connect to any server.

\begin{lstlisting}
skv_status_t Finalize();
\end{lstlisting}

Finalize will cleanup any initialized and accumulated state of the SKV
client.


\subsection{Server Connection}\label{sec:api:connect}

\begin{lstlisting}
skv_status_t Connect( const char* aConfigFile,
                      int aFlags );
\end{lstlisting}

Before a client can access a server, it has to connect to a server.
\begin{description}
\item[aConfigFile] Since the server may consist of a group of
  processes, it's not meaningful to connect to a particular address.
  Instead, the config file contains the name and the path of the
  server's machinefile which contains a list of addresses and port
  numbers for the client to connect to.  This might change in the
  future, especially, since the config file was already provided with
  the Init() call (see Section\,\ref{sec:api:init}).
\item[aFlags] Connection flags for SKV. Currently unused.
\end{description}

NOTE: The parameters for this are under review.  Especially, the
config file parameter might change into a single address again. For
example, one could think of connecting to one server and retrieve the
list of server addresses from it.


\begin{lstlisting}
skv_status_t Disconnect();
\end{lstlisting}

Disconnecting from server(s).  Notice, the lack of distinction between
different server connections, means that one client object can have a
connection to only one server group at a time.  If an application
needs to connect to different servers, it has to use multiple client
objects that are initialized differently.



\subsection{Partitioned Data Set Operations}\label{sec:api:pds_ops}

\begin{lstlisting}
skv_status_t Open( char* aPDSName,
                   skv_pds_priv_t aPrivs,
                   skv_cmd_open_flags_t aFlags,
                   skv_pds_id_t* aPDSId );

skv_status_t iOpen( char* aPDSName,
                    skv_pds_priv_t aPrivs,
                    skv_cmd_open_flags_t aFlags,
                    skv_pds_id_t* aPDSId,
                    skv_client_cmd_ext_hdl_t* aCmdHdl );
\end{lstlisting}

Open access to a Partitioned Data Set (PDS).  There's a blocking and a
non-blocking version of the Open() call.
\begin{description}
\item[aPDSName] A \code{char*} representation of the PDS.  Each PDS
  needs to have a name represented by a c-style string.

\item[aPrivs] The privileges that will be used while the PDS is open.
  \begin{description}
  \item[\tiny SKV\_PDS\_READ] Allow users to read data from PDS.
  \item[\tiny SKV\_PDS\_WRITE] Allow users to write data to the PDS.
  \end{description}

\item[aFlags] A set of flags determining the response of the server
  under certain conditions.
  \begin{description}
  \item [\tiny SKV\_COMMAND\_OPEN\_FLAGS\_NONE] No special flags. This is
    the default.
  \item[\tiny SKV\_COMMAND\_OPEN\_FLAGS\_CREATE] Create a new PDS if it
    doesn't exists.
  \item[\tiny SKV\_COMMAND\_OPEN\_FLAGS\_EXCLUSIVE] Only try to open the PDS
    if it exists.  Otherwise return an error.
  \item[\tiny SKV\_COMMAND\_OPEN\_FLAGS\_DUP] Allow opening of a PDS that
    has duplicates. (Not fully supported at the moment)
  \end{description}

\item[aPDSId] Open will return a PDSId which is required for any
  further interaction with this PDS.  Inserting or retrieving data,
  everything needs the returned PDSId.

\item[aCmdHdl] If the non-blocking version of the command is used,
  this will return the command handle to check for completion of the
  operation.
\end{description}


\begin{lstlisting}
skv_status_t PDScntl( skv_pdscntl_cmd_t aCmd,
                      skv_pds_attr_t *aPDSAttr );

skv_status_t iPDScntl( skv_pdscntl_cmd_t aCmd,
                       skv_pds_attr_t *aPDSAttr,
                       skv_client_cmd_ext_hdl_t *aCmdHdl );
\end{lstlisting}

Perform various commands for a PDS.  This allows to retrieve
information about an existing PDS.  For example to return the
attributes.
\begin{description}
\item[aCmd] The command to execute.
  \begin{description}
    \item[\tiny SKV\_PDSCNTL\_CMD\_STAT\_SET] Set the attributes of a
      PDS.
    \item[\tiny SKV\_PDSCNTL\_CMD\_STAT\_GET] Get the attributes of a
      PDS.
    \item[\tiny SKV\_PDSCNTL\_CMD\_CLOSE] Close a PDS.  There's a
      separate Close() call available which has a simpler interface.
  \end{description}

\item[aPDSAttr] The attribute argument serves as both, input and
  output parameter.  The input has at least to be filled with the PDS
  name or the PDSid in order to find the PDS for the command.

\item[aCmdHdl] If the non-blocking version of the command is used,
  this will return the command handle to check for completion of the
  operation.
\end{description}


\begin{lstlisting}
skv_status_t Close( skv_pds_id_t* aPDSId );

skv_status_t iClose( skv_pds_id_t* aPDSId,
                     skv_client_cmd_ext_hdl_t* aCmdHdl );
\end{lstlisting}

Close an open PDS with a given PDSId.
\begin{description}
\item[aPDSId] The id of the PDS to close.  This id is returned by Open().
\item[aCmdHdl] If the non-blocking version of the command is used,
  this will return the command handle to check for completion of the
  operation.
\end{description}




\subsection{Inserting KV-Pairs}\label{sec:api:insert}

\begin{lstlisting}
skv_status_t Insert( skv_pds_id_t* aPDSId,
                     char* aKeyBuffer,
                     int aKeyBufferSize,
                     char* aValueBuffer,
                     int aValueBufferSize,
                     int aValueBufferOffset,
                     skv_cmd_RIU_flags_t aFlags );

skv_status_t iInsert( skv_pds_id_t* aPDSId,
                      char* aKeyBuffer,
                      int aKeyBufferSize,
                      char* aValueBuffer,
                      int aValueBufferSize,
                      int aValueBufferOffset,
                      skv_cmd_RIU_flags_t aFlags,
                      skv_client_cmd_ext_hdl_t* aCmdHdl );
\end{lstlisting}

Inserts a key-value pair to a PDS.
\begin{description}
\item[aPDSId] The PDSId returned from the Open() call.
\item[aKeyBuffer] Pointing to a buffer that contains the key.
\item[aKeyBufferSize] The size of the key.  Note, the maximum size of
  the key is currently limited by the size of the control message
  (\code{SKV\_CONTROL\_MESSAGE\_SIZE} defined in file
  include/common/skv\_types.hpp.
\item[aValueBuffer] Pointing to the buffer that contains the value.
\item[aValueBufferSize] The size of the value to insert.
\item[aValueBufferOffset] The offset at which the value buffer will be
  inserted at the server.  The offset has no effect on the source
  value buffer.
\item[aFlags] Insert flags described in Section\,\ref{sec:api:RIU:flags}.
\item[aCmdHdl] If the non-blocking version of the command is used,
  this will return the command handle to check for completion of the
  operation.
\end{description}


\subsection{Update KV-Pairs}\label{sec:api:update}
\begin{lstlisting}
skv_status_t Update( skv_pds_id_t* aPDSId,
                     char* aKeyBuffer,
                     int aKeyBufferSize,
                     char* aValueBuffer,
                     int aValueUpdateSize,
                     int aOffset,
                     skv_cmd_RIU_flags_t aFlags );

skv_status_t iUpdate( skv_pds_id_t* aPDSId,
                      char* aKeyBuffer,
                      int aKeyBufferSize,
                      char* aValueBuffer,
                      int aValueUpdateSize,
                      int aOffset,
                      skv_cmd_RIU_flags_t aFlags,
                      skv_client_cmd_ext_hdl_t* aCmdHdl );
\end{lstlisting}
This command is new and not yet implemented at the server. The
interface might change in the near future to include update triggers.


\subsection{Retrieve KV-Pairs}\label{sec:api:retrieve}
\begin{lstlisting}
skv_status_t Retrieve( skv_pds_id_t* aPDSId,
                       char* aKeyBuffer,
                       int aKeyBufferSize,
                       char* aValueBuffer,
                       int aValueBufferSize,
                       int* aValueRetrievedSize,
                       int aOffset,
                       skv_cmd_RIU_flags_t aFlags );

skv_status_t iRetrieve( skv_pds_id_t* aPDSId,
                        char* aKeyBuffer,
                        int aKeyBufferSize,
                        char* aValueBuffer,
                        int aValueBufferSize,
                        int* aValueRetrievedSize,
                        int aOffset,
                        skv_cmd_RIU_flags_t aFlags,
                        skv_client_cmd_ext_hdl_t* aCmdHdl );
\end{lstlisting}
Retrieves the value (or parts) for a given key.
\begin{description}
\item[aPDSId] The PDSId returned from the Open() call.
\item[aKeyBuffer] Pointing to a buffer that contains the key.
\item[aKeyBufferSize] The size of the key.  Note, the maximum size of
  the key is currently limited by the size of the control message
  (\code{SKV\_CONTROL\_MESSAGE\_SIZE} defined in file
  include/common/skv\_types.hpp.
\item[aValueBuffer] Pointing to the buffer that will hold the value
  after completion.
\item[aValueBufferSize] The number of bytes that the user requests to
  retrieve from the server.  Needs to be less than or equal to the
  size of the ValueBuffer.
\item[aValueRetrievedSize] Returns the actual number of retrieved
  bytes if the parameter is not NULL and the return status is
  \code{SKV\_SUCCESS}
\item[aOffset] The offset in the stored value where the retrieve
  should start to fetch data into \code{aValueBuffer}.
\item[aFlags] Retrieve flags. (See Section\,\ref{sec:api:RIU:flags})
\item[aCmdHdl] If the non-blocking version of the command is used,
  this will return the command handle to check for completion of the
  operation.
\end{description}


% \paragraph{Retrieve Size and Return Code}
% The returned data size and error code maybe counter-intuitive.
% However, this way it allows the retrieve command to return more
% information about the stored value at the server.  This extended
% semantics of parameter \code{ValueRetrievedSize} will return
% \code{SKV\_ERRNO\_VALUE\_TOO\_LARGE} if the data at the server is
% larger than the requested size.  In this situation, the data is
% properly retrieved into the user buffer and the retrieved size matches
% the requested size.  It's only that the return value in
% \code{ValueRetrievedSize} reflects the total size of the stored value
% and therefore will be larger than \code{ValueBufferSize}.

\subsection{Flags for Insert, Retrieve, and Update}\label{sec:api:RIU:flags}
Explanation of insert flags:  \todo{ls: need check for implementation
  inconsistencies of some flags}
\begin{description}
\item[SKV\_COMMAND\_RIU\_FLAGS\_NONE] This is the default.  If a
  record with the same key already exists, SKV returns an error
  (\code{SKV\_ERRNO\_RECORD\_ALREADY\_EXISTS}).
\item[SKV\_COMMAND\_RIU\_APPEND] If a record with the same key
  already exists, the new value is appended at the end of the existing
  value(s).
\item[SKV\_COMMAND\_RIU\_UPDATE] If a record with the same key already
  exists, the existing value is overwritten with the new one.  If the
  new value has a different size, the size is adjusted to match the
  lenght of the new value.  This also includes partial updates with
  provided offsets as long as $offset <= existing\_value\_size$.
\item[SKV\_COMMAND\_RIU\_INSERT\_EXPANDS\_VALUE] When used with the
  proper offset, this flag can be used to append new data at or after
  the end of existing values.  The semantics is the same as with
  append except that the user is responsible for using correct
  offsets.  Note that the provided offset has to be larger or equal to
  the size of the existing value (non-overlapping inserts).
\item[SKV\_COMMAND\_RIU\_INSERT\_OVERWRITE\_VALUE\_ON\_DUP] Overwrites
  existing parts of the value without adjusting the size of the
  record.  Only works properly if $offset + length <= size$.
\item[SKV\_COMMAND\_RIU\_INSERT\_OVERLAPPING] Allow overlapping
  updates. For example in conjunction with
  \code{SKV\_COMMAND\_RIU\_INSERT\_EXPANDS\_VALUE} to enable arbitrary
  partial updates to a value.
\item[SKV\_COMMAND\_RIU\_RETRIEVE\_SPECIFIC\_VALUE\_LEN] Enable report of
  the totalsize even if less data was requested.
\end{description}


\subsection{Remove KV-Pairs}\label{sec:api:remove}
\begin{lstlisting}
skv_status_t Remove( skv_pds_id_t* aPDSId,
                     char* aKeyBuffer,
                     int aKeyBufferSize,
                     skv_cmd_remove_flags_t aFlags );

skv_status_t iRemove( skv_pds_id_t* aPDSId,
                      char* aKeyBuffer,
                      int aKeyBufferSize,
                      skv_cmd_remove_flags_t aFlags,
                      skv_client_cmd_ext_hdl_t* aCmdHdl );
\end{lstlisting}
Deletes a Key/Value pair from the server storage.
\begin{description}
\item[aPDSId] The PDSId returned from the Open() call.
\item[aKeyBuffer] Pointing to a buffer that contains the key.
\item[aKeyBufferSize] The size of the key.  Note, the maximum size of
  the key is currently limited by the size of the control message
  (\code{SKV\_CONTROL\_MESSAGE\_SIZE} defined in file
  include/common/skv\_types.hpp.
\item[aFlags] Command flags for remove. \todo{check if used at all}
\item[aCmdHdl] If the non-blocking version of the command is used,
  this will return the command handle to check for completion of the
  operation.
\end{description}


\subsection{Bulk Insert}\label{sec:api:bulkinsert}
Allows accumulated insertion of data.

\begin{lstlisting}
skv_status_t CreateBulkInserter( skv_pds_id_t* aPDSId,
                                 skv_bulk_inserter_flags_t aFlags,
                                 skv_client_bulk_inserter_ext_hdl_t* aBulkInserterHandle );
\end{lstlisting}
Creates a context/handle for bulk insertion of data.
\begin{description}
\item[aPDSId] The PDSId returned from the Open() call.
\item[aFlags] Bulk-inserter flags for configuration.
\item[aBulkInserterHandle] returned context/handle for use with
  insert, flush, and close operations.
\end{description}

\begin{lstlisting}
skv_status_t Insert( skv_client_bulk_inserter_ext_hdl_t aBulkInserterHandle,
                     char* aKeyBuffer,
                     int aKeyBufferSize,
                     char* aValueBuffer,
                     int aValueBufferSize,
                     skv_bulk_inserter_flags_t aFlags
                     );
\end{lstlisting}
Add a key/value pair into the bulk-inserter.  If the internal buffer
reaches a threshold, it will be automatically flushed. Otherwise, the
Flush() command can be used.
\begin{description}
\item[aBulkInserterHandle] The context/handle returned from the
  CreateBulkInserter() call.
\item[aKeyBuffer] Pointing to a buffer that contains the key.
\item[aKeyBufferSize] The size of the key.  Note, the maximum size of
  the key is currently limited by the size of the control message
  (\code{SKV\_CONTROL\_MESSAGE\_SIZE} defined in file
  include/common/skv\_types.hpp.
\item[aValueBuffer] Pointing to the buffer that will hold the value
  after completion.
\item[aValueBufferSize] The number of bytes that the user requests to
  retrieve from the server.  Needs to be less than or equal to the
  size of the ValueBuffer.
\item[aFlags] Only \code{SKV\_BULK\_INSERTER\_FLAGS\_NONE} is
  defined yet.
\end{description}

\begin{lstlisting}
skv_status_t Flush( skv_client_bulk_inserter_ext_hdl_t aBulkInserterHandle );
\end{lstlisting}
\begin{lstlisting}
skv_status_t CloseBulkInserter( skv_client_bulk_inserter_ext_hdl_t aBulkInserterHandle );
\end{lstlisting}

  
\subsection{Asynchronous Completion}\label{sec:api:async}
\begin{lstlisting}
skv_status_t TestAny( skv_client_cmd_ext_hdl_t* aCmdHdl );

skv_status_t Test( skv_client_cmd_ext_hdl_t aCmdHdl );
\end{lstlisting}
Non-blocking test if a specific \code{aCmdHdl} is complete or not.
\code{TestAny} will return a valid handle if any outstanding request
is complete.

Once a test is successful, there's no further need for cleanup by the
application.  That means, a particular \code{aCmdHdl} will only show
up once whether the application uses test or wait variants of the
calls.

\begin{lstlisting}
skv_status_t WaitAny( skv_client_cmd_ext_hdl_t* aCmdHdl );

skv_status_t Wait( skv_client_cmd_ext_hdl_t aCmdHdl );
\end{lstlisting}
Blocking test if a specific \code{aCmdHdl} is complete or not.  The
call blocks until either any (\code{WaitAny()}) or a particular
(\code{Wait()}) command completed.




\subsection{Iterators/Cursors}\label{sec:api:cursor}
Cursors enable iteration over the existing key/value pairs of a given
PDS.  There are 2 flavors of iterators\footnote{The terms cursor and
  iterator will be used as synonyms here.}: a local and a global
cursor.

The main mode of operation for both flavors of the cursors:
\begin{enumerate}
\item Open a cursor to get a handle for further operation
\item Call \verb|GetFirst(Local)Element()| to seek to the beginning of
  the list of entries.
\item Call \verb|GetNext(Local)Element()| to iterate
\item Close the cursor to release resources and invalidate the handle.
\end{enumerate}


\subsubsection{Local Iterator}\label{sec:api:cursor_local}
The local cursor allows iteration over the key/value entries of only
one server.  The term \emph{local} might be misleading because the
client library will contact the server with a given MPI rank.  This
may be non-intuitive especially if servers and clients run on disjoint
sets of nodes/machines.

\begin{lstlisting}
skv_status_t OpenLocalCursor( int aNodeId,
                              skv_pds_id_t* aPDSId,
                              skv_client_cursor_ext_hdl_t* aCursorHdl );
\end{lstlisting}
\begin{description}
\item[aNodeId] Provide the NodeID or MPI-rank of the server you want
  to iterate.
\item[aPDSId] The PDSId returned from the Open() call.
\item[aCursorHdl] This output parameter will return a handle for the
  iterator to be used for the \verb|GetLocalNNN()| calls.
\end{description}

\begin{lstlisting}
skv_status_t GetFirstLocalElement( skv_client_cursor_ext_hdl_t aCursorHdl,
                                   char* aRetrievedKeyBuffer,
                                   int* aRetrievedKeySize,
                                   int aRetrievedKeyMaxSize,
                                   char* aRetrievedValueBuffer,
                                   int* aRetrievedValueSize,
                                   int aRetrievedValueMaxSize,
                                   skv_cursor_flags_t aFlags );

skv_status_t GetNextLocalElement( skv_client_cursor_ext_hdl_t aCursorHdl,
                                  char* aRetrievedKeyBuffer,
                                  int* aRetrievedKeySize,
                                  int aRetrievedKeyMaxSize,
                                  char* aRetrievedValueBuffer,
                                  int* aRetrievedValueSize,
                                  int aRetrievedValueMaxSize,
                                  skv_cursor_flags_t aFlags );
\end{lstlisting}

\begin{description}
\item[aCursorHdl] The cursor handle returned from the
  OpenLocalCursor() call.
\item[aRetrievedKeyBuffer] Points to a buffer that will contain the
  first retreived key.
\item[aRetrievedKeySize] Will contain the size of the retrieved key
  when the call returns.
\item[aRetrievedKeyMaxSize] The maximum size of any key expected for
  the cursor (\abrIE the size of the provided buffer).
\item[aRetrievedValueBuffer] Points to a buffer that will contain the
  first retrieved value.
\item[aRetrievedValueSize] Will contain the size of the retrieved
  value when the call returns.
\item[aRetrievedValueMaxSize] The maximum size of any value expected
  for the cursor (\abrIE the size of the provided buffer).
\item[aFlags] Currently not in use.  First- and Next- call set these
  flags accordingly so that there's nothing to do for the user.
\end{description}

\begin{lstlisting}
skv_status_t CloseLocalCursor( skv_client_cursor_ext_hdl_t aCursorHdl );
\end{lstlisting}

Closes the cursor and invalidates the cursor handle.
\begin{description}
\item[aCursorHdl] The cursor handle returned from the
  OpenLocalCursor() call.
\end{description}

\subsubsection{Global Iterator}\label{sec:api:cursor_global}
The global iterator will iterate over all available key/value pairs of
a PDS on all connected servers.  The usage of the API calls is the
same as for the local generator.

\begin{lstlisting}
skv_status_t OpenCursor( skv_pds_id_t* aPDSId,
                         skv_client_cursor_ext_hdl_t* aCursorHdl );

skv_status_t CloseCursor( skv_client_cursor_ext_hdl_t aCursorHdl );

skv_status_t GetFirstElement( skv_client_cursor_ext_hdl_t aCursorHdl,
                              char* aRetrievedKeyBuffer,
                              int* aRetrievedKeySize,
                              int aRetrievedKeyMaxSize,
                              char* aRetrievedValueBuffer,
                              int* aRetrievedValueSize,
                              int aRetrievedValueMaxSize,
                              skv_cursor_flags_t aFlags );

skv_status_t GetNextElement( skv_client_cursor_ext_hdl_t aCursorHdl,
                             char* aRetrievedKeyBuffer,
                             int* aRetrievedKeySize,
                             int aRetrievedKeyMaxSize,
                             char* aRetrievedValueBuffer,
                             int* aRetrievedValueSize,
                             int aRetrievedValueMaxSize,
                             skv_cursor_flags_t aFlags );
\end{lstlisting}


\subsection{Example Client}\label{sec:api:example}
Source not pasted here.  There are several examples that can be found
in the SKV source tree under \verb|test|.

\endinput



%%% Local Variables: 
%%% mode: latex
%%% TeX-master: "skvdoc"
%%% End: 

%%%%%%%%%%%%%%%%%%%%%%%%%%%%%%%%%%%%%%%%
% Copyright (c) IBM Corp. 2014
% All rights reserved. This program and the accompanying materials
% are made available under the terms of the Eclipse Public License v1.0
% which accompanies this distribution, and is available at
% http://www.eclipse.org/legal/epl-v10.html
%
% Contributors:
%    lschneid
%%%%%%%%%%%%%%%%%%%%%%%%%%%%%%%%%%%%%%%%

\section{Configuration}\label{sec:skv:config}
This section will cover the compiletime and runtime configuration of
SKV client and server.

\subsection{Compile Time Configuration}\label{sec:skv:config:cmp}
There are a ton of macros that affect the behaviour and performance of
SKV.  Most of them are bundled in \verb|make.conf.in|.

When SKV source is build for the first time, the \verb|bootstrap.sh|
script will be executed.  It creates an initial \verb|make.conf| by
copying the original \verb|make.conf.in|.  After this step the default
configuration is set up.

For more system and environment specific configurations like paths and
compiler flags, SKV provides an option to do individual configuration
without modifying the main config.  You'll find a few files named
\verb|make.conf_dev| or \verb|make.conf_rpm|.  These files are
prepended to \verb|make.conf| and selected via environment variable
\verb|SKV_CONF|.  For example:

\begin{verbatim}
$> export SKV_CONF=dev
\end{verbatim}
 
Will select \verb|make.conf_dev| to be used to compile SKV.  If any of
the existing files is close to what you plan to do, make your own copy
and set \verb|SKV_CONF| accordingly.


\paragraph{IT\_API Configuration} works the same way as the SKV
configuration.  Since the IT\_API component was moved into SKV only
recently, the build and configuration is not yet fully merged.
Therefore, it still has parts of the configuration in separate files
in the \verb|it_api| directory.



\subsection{Client Libraries}\label{sec:config:libs}
SKV supports MPI-based and non-MPI clients.  Both have some slight
differences in the API, especially for initialization.  This is also
reflected by the availability of 2 flavors of libraries that need to
be linked to a client application.

The \verb|libskv_common| and \verb|libskv_client| have a non-MPI
version and a MPI-version.  The MPI-version includes \verb|_mpi| in
the names.

To build a non-MPI client, you need to define the macro
\verb|SKV_CLIENT_UNI| before including
\verb|include/client/skv_client.hpp|.


There's also a C-style library for SKV named \verb|libskvc.a| (and
\verb|libskvc_mpi.a| for MPI clients).  The C include file is
\verb|lib/skv.h|.


\subsection{Runtime Configuration}\label{sec:skv:config:run}
The main configuration of SKV is done via the \verb|skv_server.conf|
file.  Despite its name, it's also used by the client to find out
about the server location and connection options.

An example server config file can be found in
\verb|system/bgq/etc/skv_server.conf|.  The comments in this file
explain the details of each setting.  However, for convenience,
settings are explained below:
\begin{description}
\item[SKV\_SERVER\_PORT] will be the base port number the server will
  start trying to listen.  If the port is blocked by another SKV
  server instance (\abrEG if multiple servers run on the same node),
  SKV will attempt to bind to the subsequent ports.  The limit for
  attempts is determined by the macro \verb|SKV_MAX_SERVER_PER_NODE|
  in file \verb|include/common/skv_types.hpp|.
\item[SKV\_SERVER\_READY\_FILE] determines the file name that the
  server will create when it's ready to accept connections from
  clients.  The file is an empty file.
\item[SKV\_SERVER\_MACHINE\_FILE] This file will contain the IP
  addresses/hostnames and port numbers of the running server.  This
  type of file is required for a client to connect to a server.
\item[PERSISTENT\_FILENAME] This is the base file name of the file or
  directory used by the SKV server for persisting the data.  For
  example, in case of the in-memory storage back-end, this will be the
  base name of the mmapped file.
\item[PERSISTENT\_FILE\_LOCAL\_PATH] represents the full absolute path
  and base name for the persistent file or directory.  The server will
  append a rank number and maybe other identifiers to make this file
  unique within a server group.
\item[SKV\_SERVER\_LOCAL\_INFO\_FILE] In case SKV is exclusively run
  on a single node with multiple instances, this file is used instead
  of the server machine file.  It is suggested to use the same name
  here as for \verb|SKV_SERVER_MACHINE_FILE| because this option is
  obsolete and might disappear in the future.
\item[SKV\_SERVER\_COMM\_IF] Determines the primary interface that's
  picked by the server.  The IP address or hostname of this device
  will be placed in the machine file.
\end{description}


\subsubsection{Running the Server}\label{sec:skv:run:server}
When the server is started, the following options for the config file
are available:
\begin{enumerate}
\item an absolute path and file name provided at the command line via
  \verb|-c| option.
\item A user specific config file \verb|$(HOME)/.skv_server.conf|
\item A system-wide config file \verb|/etc/skv_server.conf|
\item Compile-time defaults will be used if there's no config file
  found or provided by any of the above options.
\end{enumerate}



\subsubsection{Running a Client}\label{sec:skv:run:client}
During the initialization of a client, the configuration file is
looked up via:
\begin{enumerate}
\item A user specific config file \verb|$(HOME)/.skv_server.conf|
\item A system-wide config file \verb|/etc/skv_server.conf|
\item Compile-time defaults will be used if there's no config file
  found or provided by any of the above options.
\end{enumerate}




\endinput



%%% Local Variables: 
%%% mode: latex
%%% TeX-master: "skvdoc"
%%% End: 

%%%%%%%%%%%%%%%%%%%%%%%%%%%%%%%%%%%%%%%%
% Copyright (c) IBM Corp. 2014
% All rights reserved. This program and the accompanying materials
% are made available under the terms of the Eclipse Public License v1.0
% which accompanies this distribution, and is available at
% http://www.eclipse.org/legal/epl-v10.html
%
% Contributors:
%    lschneid
%%%%%%%%%%%%%%%%%%%%%%%%%%%%%%%%%%%%%%%%

\section{Internal API for Storage Back-end (tbd)}\label{sec:intapi}

Description of the API between the SKV state machine and the storage
back-end.  Explains the requirements and semantics of the interaction
for both, the back-end call interface and the event-based response
option for the storage back-end.

tbd\dots

\endinput



%%% Local Variables: 
%%% mode: latex
%%% TeX-master: "skvdoc"
%%% End: 

%%%%%%%%%%%%%%%%%%%%%%%%%%%%%%%%%%%%%%%%
% Copyright (c) IBM Corp. 2014
% All rights reserved. This program and the accompanying materials
% are made available under the terms of the Eclipse Public License v1.0
% which accompanies this distribution, and is available at
% http://www.eclipse.org/legal/epl-v10.html
%
% Contributors:
%    lschneid
%%%%%%%%%%%%%%%%%%%%%%%%%%%%%%%%%%%%%%%%

\section{Command Paths}\label{sec:skv:commands}

\subsection{Init}
\code{Init()} is the first routine to call when creating a SKV
client.
\begin{enumerate} \parskip-0.5ex
\item \code{skv\_client\_t::Init()}
\item \code{skv\_client\_internal\_t::Init()}
  \begin{enumerate} \parskip-0.5ex
  \item creates IT\_API device
  \item creates IT\_API protection zone
  \item initialize Command Control Block Manager Interface
  \item initialize Connection Manager Interface
  \item initialize Command Manager Interface
  \item initialize Relational Cursor Manager Interface
  \end{enumerate}
\end{enumerate}


\subsection{Connect}
Connect to a SKV server.
\begin{enumerate} \parskip-0.5ex
\item \code{skv\_client\_t::Connect()}
\item \code{skv\_client\_internal\_t::Connect()}
  \begin{enumerate} \parskip-0.5ex
  \item \code{skv\_client\_conn\_manager\_if\_t::Connect()}
    \begin{enumerate} \parskip-0.5ex
    \item gets client hostname via \code{gethostname} to find out
      where it is running (determine method to get server IP)
    \item Get \code{ComputeFileNamePath} with server IPs
    \item Number of SKV servers determined by number of lines in\\ \code{ComputeFileNamePath}
    \item Fetches server addresses/names from file
    \item Chose a random server to connect first, then connect
      subsequent servers (\code{ConnectToServer()})
      \begin{enumerate}
      \item creates RC endpoint
      \item gets IPv4 address via \code{gethostbyname()}
      \item does \code{it\_ep\_connect()}
      \item polls the event dispatchers for connection manager msgs
        (and affiliated events and unaffiliated events for errors)
      \end{enumerate}
    \item Retrieves the distribution function
      \begin{enumerate}
      \item Creates a command

        calls \code{skv\_client\_command\_manager\_if\_t::Reserve()}
        which removes an element from queue of free Command Control
        Blocks (CCB). The queue is implemented as a double linked
        list and implements a stack semantic (LIFO).
      \item initializes a \code{skv\_cmd\_retrieve\_dist\_req\_t}
      \item dispatch the command to a random server
      \item wait for the command to complete
      \end{enumerate}
    \end{enumerate}
  \end{enumerate}
\end{enumerate}


\subsection{Open}
Create or open a partitioned data set (PDS).  Open uses \code{iOpen} and \code{wait}.
\begin{itemize}
\item Use the PDS-name as a key and decide the server number of the owner
\end{itemize}




\section{The SKV Server State Machine}\label{sec:skv:srv_sm}

Questions to answer in this section:
\begin{itemize}
\item Which server states are defined
\item Event types that cause which transition
\item Which events are triggered when transitioning
\end{itemize}


\section{Implementation}\label{sec:skv:impl}

\subsection{Endpoint State}
An Endpoing State (EPState) holds connection-relevant information and
keeps track of QP resources.

\emph{Architecturally this might be questionable. A QP status is
  required to maintain. However, since the main server state machine
  multiplexes all events through a single aggregated event queue, it
  requires another demultiplexing step to get into the EPStates and
  hold their information... Maybe this is the best solution already. }



\subsection{Command Control Block}
A command control block (CCB) is assigned to each command/request that is
\emph{currently processed} or \emph{pending}. It contains the required
information about the command and its status.
An essential attribute is the \emph{command ordinal}.

There's a stack of available CCBs for each EPState.  The following
steps describe the CCB handling:
\begin{enumerate}
\item A free CCB is assigned to an event at the time of
  recv-completion for a new request.  This happens before the
  initialization of the event (routine \code{GetITEvent()}).
\item The command ordinal of the new CCB is assigned to the event to
  be able to find the correct CCB later.
\item Processing of the command
\item The CCB is returned to the pile of free slots after successfull
  dispatching of the response to the client.  Note: since the involved
  send and recv buffers are detached from the CCB, this return already
  happens after send and recv operations are posted regardless of
  their completion status.  For send and recv buffers see below.
\end{enumerate}


\subsection{Command Send Buffers}
These buffers hold the data that is sent back to the client.  They
have to be preserved until the send completion event is processed and
therefore are detached from the handling of the CCB.

An available send buffer is always attached to a single CCB.  The send
buffer is replaced by a spare after the response to the client is
posted and before the CCB is returned to the free command slots list.
Replacing the send buffer means to get an unposted buffer from a stack
of buffers.

A posted buffer is returned to the stack if the corresponding send
operation is completed (send completion event).



\subsection{Command Recv Buffers}
Recv buffers provide space to receive requests from a client.  They
have been detached from the CCB because of a potential race
condition.

As a new request arrives (in a receive buffer), this buffer gets
assigned/attached to a free CCB.  This assignment is kept until the
command is processed and the response is dispatched.  Right before
posting a new recv for future commands, the old recv buffer is
detached from the CCB and 


\subsection{Client-Server Protocol}

Header data consists of the command type (Insert, Retrieve, Remove,
...), a command ordinal, and the address of a command control block.
The command ordinal and the CCB information is primarily required by
the client to pick the right client CCB after receiving the response
from the server.


At the server, the header data is generally kept in the receive buffer
that is attached to the CCB.  An exception is if the command requires
2 phases, \abrIE RDMA-read in case of the insert command.  The
header information is copied from the receive buffer to the
CommandState member of the CCB for later use.




\endinput



%%% Local Variables: 
%%% mode: latex
%%% TeX-master: "skvdoc"
%%% End: 

\end{document}



%%% Local Variables: 
%%% mode: latex
%%% End: 
